\documentclass[a4paper,11pt]{book}
\usepackage[T1]{fontenc}
\usepackage[utf8]{inputenc}
\usepackage{lmodern}
%\usepackage{expl3}
%\expandafter\def\csname ver@l3regex.sty\endcsname{}

\newenvironment{definition}[2]{ % word, category
  %\begin{minipage}{\dimexpr\textwidth}
  \begin{description}
  %\item[#1] \textit{(#2)}:
  \item
    {\huge \textbf{#1}}
    {\Large \textit{(#2)}} \\
}{
  \end{description}%
  %\end{minipage}%
}

\newenvironment{example}{ % put example sentences in here
  \item
  \textbf{nimi lon toki:}
  
  \hfill
  \begin{minipage}{\dimexpr\textwidth-1cm}
  %\begin{quote}
  \begin{itshape}
}
{
  \end{itshape}
  \end{minipage}
  %\end{quote}
}

\newcommand{\inex}[1]{% use this to format the word when it appears in example sentences
  \textbf{#1}%
}

\title{sona nimi}
\author{jan Salinsen}

\begin{document}

\maketitle
\tableofcontents

\chapter{nimi pu}

\begin{definition}{a}{nimi lili}
  nimi \inex{a} li pini e nimi ante la ona li wawa e ona. kin la ona li ken toki e ni: nimi ni li suli nanpa wan.
  \begin{example}
    soweli ni li suwi \inex{a}!
    
    moku sina li seli \inex{a}!
  \end{example}
\end{definition}

\begin{definition}{akesi}{nimi awen}
  \inex{akesi} mute li jo e noka tu tu e palisa monsi, taso \inex{akesi} linja li lon li jo ala e noka. \inex{akesi} ale li jo e selo nasa. ona li ken ala kama seli tan insa. seli ona li tan suno taso.
  \begin{example}
    uta pi \inex{akesi} linja ni li ken ala ken moli e mi?
    
    \inex{akesi} akesi suwi pi jan San li moku e pipi lon tenpo poka. o kama pona o lukin!
  \end{example}
\end{definition}

\begin{definition}{ala}{nimi ante}
  ijo li ken lon. ante la ijo \inex{ala} li lon. nimi \inex{ala} li ante e kon pi nimi ante sama ni: nimi \inex{ala} li pini e nimi la lon ijo li weka. ijo li poka ala la ona li weka. ijo li tawa ala la ona li awen.
  \begin{example}
    lipu sina li ike la mi sona \inex{ala} e kon pi toki ni.
    
    mi lon sona nimi nanpa tu wan taso la mi wile \inex{ala} awen pali e lipu ni.
  \end{example}
\end{definition}

\pagebreak

\begin{definition}{alasa}{nimi tawa}
  \inex{alasa} li open jo. kin la ona li ken ante e nimi pini tawa kon ni: ken sina taso la sina ijo. sina \inex{alasa} ijo la sina wile ijo. sina open ijo tan wile ni. taso sina ken pini ala ijo. kon kin pi nimi lukin li ken sama ni.
  \begin{example}
    o tawa tomo esun o \inex{alasa} e pan e sike waso.
    
    mi \inex{alasa} kama sona e toki pona taso ken mi li wawa ala.
  \end{example}
\end{definition}

\begin{definition}{ale, ali}{nimi awen}
  ijo \inex{ale} li toki e kulupu ijo ni: ijo li lon la ona li lon kulupu. sina ken kepeken nimi \inex{ale} sama ni: mi moku e kala \inex{ale}. toki ni la kala pi moku ala sina li lon ala. jan mute li kepeken nimi \inex{ali} kin. nimi tu ni li nimi wan. ante sitelen en ante kalama li tan ni taso: kalama pi nimi \inex{ale} li ken sama ike kalama pi nimi ala.
  \begin{example}
    sina \inex{ale} o! seme li lon?
    
    len \inex{ale} mi li jaki la mi o telo e ona.
  \end{example}
\end{definition}

\begin{definition}{anpa}{nimi awen}
  nimi \inex{anpa} li toki e wile sona ni: ijo li lon seme? ijo li lon \inex{anpa} la ona li lon poka noka li lon poka ma. ona li sama pipi linja tawa waso. nimi ni li ken toki e ni kin: ijo li wawa ala. sina \inex{anpa} e ijo la sina weka e wawa ona kepeken wawa sina.
  \begin{example}
    pakala! jaki soweli li lon \inex{anpa} pi len noka mi.
    
    jan Emili en jan Sijewa li utala! jan Sijewa o \inex{anpa} e jan Emili.
  \end{example}
\end{definition}
\end{document}
