\documentclass[a4paper,11pt]{book}
\usepackage[T1]{fontenc}
\usepackage[utf8]{inputenc}
\usepackage{lmodern}

\newenvironment{definition}[2]{ % word, category
  \begin{minipage}{\dimexpr\textwidth}
  \begin{description}
  \item
    {\huge \textbf{#1}}
    {\Large \textit{(#2)}} \\
}{
  \end{description}
  \end{minipage}
}

\newenvironment{example}{ % put example sentences in here
  \item
  \textbf{nimi lon toki:}

  \hfill
  \begin{minipage}{\dimexpr\textwidth-1cm}
  \begin{itshape}
}
{
  \end{itshape}
  \end{minipage}
}

\newcommand{\inex}[1]{% use this to format the word when it appears in example sentences
  \textbf{#1}%
}

\title{sona nimi}
\author{jan Salinsen}

\begin{document}

\maketitle
\tableofcontents

\chapter{nimi pu}

\setlength\parindent{0pt}

\begin{definition}{a}{nimi lili}
  nimi \inex{a} li pini e nimi ante la ona li wawa e ona. kin la ona li ken toki e ni: nimi ni li suli nanpa wan.
  \begin{example}
    soweli ni li suwi \inex{a}!
    
    moku sina li seli \inex{a}!
  \end{example}
\end{definition}

\begin{definition}{akesi}{nimi suli}
  \inex{akesi} mute li jo e noka tu tu e palisa monsi, taso \inex{akesi} linja li lon li jo ala e noka. \inex{akesi} ale li jo e selo nasa. ona li ken ala kama seli tan insa. seli ona li tan suno taso.
  \begin{example}
    uta pi \inex{akesi} linja ni li ken ala ken moli e mi?
    
    \inex{akesi} suwi pi jan San li moku e pipi lon tenpo poka. o kama pona o lukin!
  \end{example}
\end{definition}

\begin{definition}{ala}{nimi suli}
  ijo li ken lon. ante la ijo \inex{ala} li lon. nimi \inex{ala} li ante e kon pi nimi ante sama ni: nimi \inex{ala} li pini e nimi la lon ijo li weka. ijo li lon poka \inex{ala} la ona li weka. ijo li tawa \inex{ala} la ona li awen.
  \begin{example}
    lipu sina li ike la mi sona \inex{ala} e kon pi toki ni.
    
    mi lon sona nimi nanpa tu wan taso la mi wile \inex{ala} awen pali e lipu ni.
  \end{example}
\end{definition}

%\pagebreak

\begin{definition}{alasa}{nimi suli}
  \inex{alasa} li open jo. kin la ona li ken ante e nimi pini tawa kon ni: ken sina taso la sina ijo. sina \inex{alasa} ijo la sina wile ijo. sina open ijo tan wile ni. taso sina ken pini ala ijo. kon kin pi nimi lukin li ken sama ni.
  \begin{example}
    o tawa tomo esun o \inex{alasa} e pan e sike waso.
    
    mi \inex{alasa} kama sona e toki pona taso ken mi li wawa ala.
  \end{example}
\end{definition}

\begin{definition}{ale}{nimi suli}
  ijo \inex{ale} li toki e kulupu ijo ni: ijo li lon la ona li lon kulupu. sina ken kepeken nimi \inex{ale} sama ni: mi moku e kala \inex{ale}. toki ni la kala pi moku ala sina li lon ala.
  \begin{example}
    sina \inex{ale} o! seme li lon?
    
    len \inex{ale} mi li jaki la mi o telo e ona.
  \end{example}
\end{definition}

\begin{definition}{ali}{nimi suli}
  lon la nimi \inex{ali} en nimi \inex{ale} li nimi wan. ante sitelen en ante kalama li tan ni taso: kalama pi nimi \inex{ale} li ken sama ike kalama pi nimi ala.
\end{definition}

\begin{definition}{anpa}{nimi suli}
  nimi \inex{anpa} li toki e wile sona ni: ijo li lon seme? ijo li lon \inex{anpa} la ona li lon poka noka li lon poka ma. ona li sama pipi linja tawa waso. nimi ni li ken toki e ni kin: ijo li wawa ala. sina \inex{anpa} e ijo la sina weka e wawa ona kepeken wawa sina.
  \begin{example}
    pakala! jaki soweli li lon \inex{anpa} pi len noka mi.
    
    jan Emili en jan Sijewa li utala! jan Sijewa o \inex{anpa} e jan Emili.
  \end{example}
\end{definition}

\begin{definition}{ante}{nimi suli}
  ijo tu li sama ala la ona li \inex{ante}. o sona e selo e kalama e suli e lon pi ijo tu. ale ni pi ijo nanpa wan en ale ni pi ijo nanpa tu li wan ala wan? ala la, ala ni li \inex{ante} ijo. kin la jan li ken toki e ni: ijo wan taso li \inex{ante}. kon li ni: ijo ni pi tenpo open en ijo ni pi tenpo pini la \inex{ante} li lon.
  \begin{example}
    sina kama sona wawa la sina ken \inex{ante} e toki Inli tawa toki pona a!
    
    len ni li pona a tawa mi! taso suli ona en suli pi sijelo mi li \inex{ante} la mi ken ala jo.
  \end{example}
\end{definition}

%\pagebreak

\begin{definition}{anu}{nimi lili}
  ijo \inex{anu} ijo ante la wan taso pi ijo tu ni li ken lon. ijo li ken. ijo ante kin li ken. taso tu li wile ala. wan taso li pona. kin la sina ken ante e toki sina tawa wile sona kepeken toki ni: ijo \inex{anu} seme? ni li sama toki ni: ijo ala ijo?
  \begin{example}
    moku la o pali e pan walo \inex{anu} waso. mi wile kepeken wan pi moku tu ni.
    
    sina toki mute a tawa meli ni. ona li pona tawa sina \inex{anu} seme?
  \end{example}
\end{definition}

\begin{definition}{awen}{nimi suli}
  ijo li \inex{awen} la ona li tawa ala. ijo li ante ala lon ante tenpo la \inex{awen} li lon. nimi \inex{awen} li ken open kin e nimi ante li ken ante e ona sama ni: jan li \inex{awen} ijo la tenpo li ante la ona li pini ala ijo. ona li ijo lon tenpo pi open ante lon tenpo pi pini ante kin.
  \begin{example}
    mi ken ala toki kepeken toki ilo. mi \inex{awen} kepeken ilo tawa.
    
    sina en soweli li tawa tan tomo sina la soweli o \inex{awen} lon linja.
  \end{example}
\end{definition}

\begin{definition}{e}{nimi lili}
  o kepeken sama ni: waso li ijo \inex{e} soweli. kon li ni: ijo li tan waso li tawa soweli. ni la soweli li ken kama ante tan ijo waso. sina ken kepeken kin nimi \inex{e} mute sama ni: waso li ijo \inex{e} soweli \inex{e} pipi \inex{e} kala. kon li ni: waso li ijo \inex{e} soweli. waso li ijo \inex{e} pipi. waso li ijo \inex{e} kala.
  \begin{example}
    o moku pona \inex{e} kala mi!
    
    mi wile esun \inex{e} ilo sin \inex{e} supa sin.
  \end{example}
\end{definition}

\begin{definition}{en}{nimi lili}
  ijo mute li pali e ijo la sina ken lili e toki sina kepeken nimi \inex{en} sama ni: jan San \inex{en} waso Tepi \inex{en} kala Lisa li moku. kon li ni: jan San li moku. waso Tepi kin li moku. kala Lisa kin li moku. ijo ale li lon poka open pi nimi li la o wan e toki kepeken nimi \inex{en}. ona li lon poka ante la o kepeken ala. kin la sina toki lili nasa la sina ken kepeken sama ni: kala \inex{en} waso. toki ni li lili li nasa taso kon li sama toki suli ni pi nasa ala: kala \inex{en} waso li lon. ni la o kepeken pona.
  \begin{example}
    mi tu \inex{en} jan Sonja o kama pona sin lon ma Towano a!
    
    tenpo seme la sina \inex{en} jan olin sina \inex{en} jan olin pi jan olin sina li wile kama pona tawa tomo mi?
  \end{example}
\end{definition}

%\pagebreak

\begin{definition}{esun}{nimi suli}
  ijo li pana tawa ijo ante la ijo ante ni li pana tawa ijo nanpa wan la pana tu ni li \inex{esun}. tenpo mute la kon pi nimi \inex{esun} li \inex{esun} mani tan ni: sina pana e mani tawa jan ante la ona li pana e ijo tawa sina. taso \inex{esun} ante kin pi mani ala li lon.
  \begin{example}
    mi kama sona e toki Kanse tan jan olin mi. ona li kama sona e toki Salaki tan mi. toki pi mi tu li \inex{esun} sona wawa a.
    
    mi wile moku taso moku li lon ala tomo mi. mi o tawa tomo \inex{esun} moku.
  \end{example}
\end{definition}

\begin{definition}{ijo}{nimi suli}
  kon pi nimi \inex{ijo} li ken ale. ale li \inex{ijo}. sina lukin e nimi \inex{ijo} la sina ken sona e seme tan toki ni? ni taso: ona li lon. taso sina ken pana e kon tawa nimi ni kepeken nimi ante same nimi suli ale. sama ni: \inex{ijo} pona en \inex{ijo} telo en \inex{ijo} waso.
  \begin{example}
    a! soweli li lukin e \inex{ijo} lon kasi. ona li lukin e seme?
    
    mi ken kute e jan lili lon poki ante. ona o lape taso ona li awen \inex{ijo} ante.
  \end{example}
\end{definition}

\begin{definition}{ike}{nimi suli}
  sina wile ala e ijo la ona li \inex{ike} tawa sina. \inex{ike} li ken tan ni: ona li pakala anu sina ken ala kepeken wawa ijo \inex{ike} tawa pali wile sina anu ona li pana e pilin ike tawa sina. taso ijo li \inex{ike} tawa sina la sina ken wile e ni: ona li weka tan poka sina anu ona li kama lon ala. ijo \inex{ike} li pona ala.
  \begin{example}
    jan lawa ale li \inex{ike} a! mama sina li jan lawa la ona kin li \inex{ike}.
    
    o moku ala e kili ni. ona li kama \inex{ike} tan tenpo. o pana tawa jaki.
  \end{example}
\end{definition}

\begin{definition}{ilo}{nimi suli}
  sina kepeken ijo la ona li \inex{ilo} tan kepeken. \inex{ilo} ijo li pona tawa pali ijo.
  \begin{example}
    mi wile awen toki taso \inex{ilo} toki mi li wile lape lon tenpo poka.
    
    sina jo e \inex{ilo} la ale li kama pali ilo tawa lukin sina.
  \end{example}
\end{definition}

%\pagebreak

\begin{definition}{insa}{nimi suli}
  nimi \inex{insa} li toki e wile sona ni: ijo li lon seme? ijo li lon \inex{insa} pi ijo ante la ijo ante li lon poka ale pi ijo nanpa wan li tomo anu poki tawa ona. sina lon \inex{insa} ijo la ken la sina ken ala weka tan \inex{insa} anu weka sina li pakala e ona sama ni: sina wile e \inex{insa} kili la o pakala e selo ona.
  \begin{example}
    \inex{insa} mi li pilin pakala. mi o tawa tomo pi pona sijelo.
    
    waso lili li awen lon \inex{insa} sike. o awen lukin! ona li ken kama lon ma lon tenpo poka.
  \end{example}
\end{definition}

\begin{definition}{jaki}{nimi suli}
  \inex{jaki} li sama ike suli. taso ijo li ken ike tan ni: sina wile ala kepeken ijo anu ona li pana e pilin ike tawa sina. ante \inex{jaki} ike li ni: ken la sina kepeken ala a \inex{jaki}. ona li lon poka sina taso la ona li ken ike e sina e sijelo sina. sijelo sina li ken kama pakala. \inex{jaki} li lon la wile wawa ni li lon: o weka e ona tan poka sina tan tomo sina.
  \begin{example}
    o pana e poki \inex{jaki} tawa poka nasin. tenpo poka la tomo tawa pi weka \inex{jaki} li kama.

    pakala a! o pilin e kon pi moku ni. ona li lon poki lete tan seme? o weka e ona. ona li \inex{jaki}.
  \end{example}
\end{definition}

\begin{definition}{jan}{nimi suli}
  \inex{jan} li jo e sijelo li ken wile e ijo li ken toki li ken sona e toki pi \inex{jan} ante. \inex{jan} ale la ni ale li lon taso ni: \inex{jan} ni li lon: ona li ken ala toki anu ona li ken ala sona e toki pi \inex{jan} ante. kin la ken la ona li ken taso ona li wile ala. \inex{jan} lili a li ken ala toki. toki li tan kama sona. kin la o sona e ni: ijo mute li jo e sijelo e wile e ken ale ni li sama \inex{jan} tawa lukin taso ona li \inex{jan} ala. nimi \inex{jan} li ike tawa ona tan ijo mute. ni la o kepeken ala nimi \inex{jan} tawa ona.
  \begin{example}
    \inex{jan} Sijosini li weka e selo tan waso li toki e ni: o lukin! \inex{jan} a!
    
    tenpo pi mute ike la mi awen lon supa mi. mi o tawa o toki tawa \inex{jan} ante.
  \end{example}
\end{definition}

%\pagebreak

\begin{definition}{jelo}{nimi suli}
  \inex{jelo} li kule. ko suwi pipi li \inex{jelo}. sijelo pi pipi ko ni kin li \inex{jelo} li pimeja. suno li lon sewi laso la ona li ken lukin walo taso tenpo suno li tawa pini la kule sewi li kama \inex{jelo} li kama loje.
  \begin{example}
    loje en pimeja li pona tawa sina a! loje en \inex{jelo} li ken moli e sijelo.
    
    mama mi li pana e kasi kule tawa ma. ni \inex{jelo} li pona nanpa wan tawa mi.
  \end{example}
\end{definition}

\begin{definition}{jo}{nimi suli}
  ijo li lon sina la sina \inex{jo} e ona. taso jan mute li toki e ni taso: ijo li lon luka sina la sina \inex{jo} e ona. toki ni la sina ken ala \inex{jo} e pilin e sona e ijo sama ni. taso ale li toki e ni: sina ken \inex{jo} e kili e waso e ijo sama ni.
  \begin{example}
    mi pilin ike. o \inex{jo} e mi!

    sina o kama lon tomo mi lon tenpo ante. mi wile pana e moku taso mi \inex{jo} ala.
  \end{example}
\end{definition}
\end{document}
